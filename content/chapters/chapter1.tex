\chapter*{\hspace*{6.5cm}Introducción}
\addcontentsline{toc}{chapter}{Introducción}

\noindent Las estrellas de neutrones son las estrellas más densas del universo: una estrella de neutrones estándar tiene una masa $M= 1.4\,M_{\odot}$ y un radio $R=10\,\text{km}$, lo cual implica que su densidad promedio es $\bar{\rho}=(2-3)\rho_0$ donde $\rho_{0}=2.3\times 10^{14} \text{g cm}^{-3}$ es la densidad de la materia nuclear en un núcleo atómico pesado. 

Estas estrellas tan densas fueron anticipadas por Landau en 1932 \cite{Yakovlev2013} y predichas por Baade y Zwicky en 1934 \cite{Baade1934} como el destino final de estrellas masivas que liberan grandes cantidades de energía en explosiones de supernova. Tolman y Oppenheimer \& Volkoff en 1939 \cite{Tolman1939,Oppenheimer1939} fueron los primeros en reconocer la necesidad de estudiar objetos tan densos como las estrellas de neutrones en el contexto de la relatividad general y derivaron la ecuación de equilibrio hidrostático para una estrella estática con simetría esférica. Después de dos décadas de intentos fallidos de observarlas y avances teóricos en el tema: la construcción de las primeras ecuaciones de estado que describían la materia nuclear densa, la predicción de superfluidez de la materia nuclear y el desarrollo de modelos de enfriamiento por emisión de neutrinos y radiación térmica. Las estrellas de neutrones fueron descubiertas finalmente por Jocelyn Bell en 1967 \cite{Hewish1968} en forma de pulsares: estrellas de neutrones rotantes altamente magnetizadas.

Desde entonces, las estrellas de neutrones se convirtieron en un laboratorio astrofísico que permite probar las teorías físicas desarrolladas en múltiples disciplinas a condiciones extremas que no son realizables en los laboratorios terrestres. Comportamientos exóticos como campos magnéticos con magnitudes muy grandes (de $10^4$ a $10^{11}$ T), opacidad a neutrinos y materia dominada por hiperones o quarks desconfinados, son exclusivos de las estrellas de neutrones y su descripción ha planteado retos teóricos importantes. El más grande de estos retos reside en encontrar la ecuación de estado (de aquí en adelante EOS) de la materia al interior de las estrellas de neutrones. 

Numerosos modelos de EOSs han sido desarrollados a lo largo de las últimas dos décadas basados en diferentes enfoques teóricos que son usados para extrapolar de manera consistente nuestro conocimiento de la materia nuclear a sistemas con densidades tan altas (el estado del arte en el tema es presentado en las revisiones \cite{Ozel2016,Oertel2017} y libros \cite{Haensel2007,Rezzolla2018}). El enfoque que más ha sido usado para (des)favorecer modelos de EOSs es el uso de observaciones para restringir los valores de la masa máxima y el radio típico de las estrellas de neutrones (ver \cite{Lattimer2019} para una revisión reciente del tema). Sin embargo, ante la posible existencia de varios modelos que satisfagan las restricciones observacionales esta metodología puede tener un alcance limitado.   

Un enfoque distinto para restringir los modelos de EOSs, inspirado en lo realizado en el campo de las soluciones exactas a las ecuaciones de Einstein en relatividad general (ver \cite{Delgaty1998} por ejemplo), es usar condiciones de regularidad, consistencia y estabilidad físicas para evaluar si los modelos estelares (soluciones a las ecuaciones de Einstein) obtenidos usando un modelo determinado de EOS, son físicamente plausibles. Recientemente uno de estos criterios fue usado satisfactoriamente para restringir EOSs en \cite{Koliogiannis2019a}.

Teniendo en cuenta lo anterior, este trabajo de grado tiene como objetivo emplear este enfoque aplicando el conjunto de condiciones de aceptabilidad reunido por B. Ivanov \cite{Ivanov2017} para el caso de un fluido anisótropo y extendido por Hernández et al. \cite{Hernandez2018}, a modelos de estrellas de neutrones estáticas obtenidos con EOSs disponibles en la literatura (se usará la elección presentada en \cite{Ozel2016}). El resultado será una exhaustiva clasificación de la aceptabilidad física de una parte de las EOSs disponibles. Esta clasificación podrá ser fácilmente verificada y extendida, ya que tanto las rutinas numéricas construidas para crear los modelos como el análisis de estos fue realizado usando software libre y se encuentra disponible como un repositorio público en Github\footnote{\url{https://github.com/DavidRamosSal/stellar_structure}}. Se corroborará además, que esta metodología es efectiva para identificar potenciales problemas con los modelos de EOSs existentes. 
%\REMARK{No sé qué citas poner aquí profe.}

Este trabajo estará organizado de la siguiente manera: en el Capítulo 1 se presentarán algunas generalidades de las estrellas de neutrones, con el fin de sintetizar parte del conocimiento que tenemos de ellas, cómo lo obtenemos y sus grandes limitaciones. En el Capítulo 2 se planteará una forma distinta de evaluar los modelos de EOSs existentes: tras mostrar cómo se crean modelos estelares de estrellas de neutrones, se enunciarán las condiciones de aceptabilidad física que los modelos estelares de cualquier modelo de EOS debe cumplir. El capítulo 3 sirve como una documentación de la rutina numérica usada para construir los modelos estelares a partir de una EOS tabulada así como del cálculo de las derivadas numéricas. En el Capítulo 4 se presentarán los resultados de aplicar las condiciones a los modelos de estrellas de neutrones obtenidos con 37 EOSs de estado realistas con distintas características y se discutirá lo encontrado para cada condición. Finalmente, en el Capítulo 5 se encuentran las conclusiones del trabajo y posibles direcciones de trabajo a futuro. Los apéndice incluido tiene como propósito presentar una deducción de la curvatura para un espacio-tiempo estático con simetría esférica usando el formalismo de Cartan así como la deducción de la solución exterior de Schwarzchild y de las ecuaciones de Tolman-Oppeinheimer-Volkoff. 

