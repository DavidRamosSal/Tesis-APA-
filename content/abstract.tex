
\begin{abstract}
    \begin{spacing}{1.0}
    \noindent \textbf{Descripción:}\\ \\
    \noindent Los objetos compactos son estrellas tan densas que el uso de la relatividad general es necesario para describir su fenomenología. Debido a que a densidades muy altas la materia está compuesta mayormente de neutrones, estos objetos son conocidos generalmente como estrellas de neutrones. La materia que conforma estas estrellas no puede ser creada en laboratorios terrestres y constituyen por lo tanto un laboratorio astrofísico ideal para verificar teorías físicas de materia densa.
    
    Modelar el interior de las estrellas de neutrones ha requerido superar una gran cantidad de retos teóricos en física nuclear. Tanto así que encontrar la ecuación de estado que gobierna la materia en tales condiciones es aún uno de los problemas abiertos de la física. Las observaciones de estrellas de neutrones permiten restringir los modelos teóricos usados para obtener la ecuación de estado. Pero la falta de precisión en las mediciones no ha permitido identificar la teoría correcta, de modo que características fundamentales como la composición precisa de la materia al interior de las estrellas de neutrones sigue en debate. 
    
    En este trabajo se propuso usar criterios de \quotes{aceptabilidad física} (consideraciones de consistencia física y estabilidad dinámica) para restringir los modelos de materia ultradensa existentes. En particular, se hizo énfasis en aplicar rigurosamente los criterios de estabilidad conocidos. Como resultado, se encontró que todos los modelos estelares encontrados con las ecuaciones de estado que se consideraron, no cumplen un simple criterio de estabilidad ante movimientos convectivos formulado recientemente por Hernández et al. \cite{Hernandez2018}.
    \end{spacing}
\end{abstract}
\newpage
\begin{abstract1}
    \begin{spacing}{1.0}
    \noindent \textbf{Description:}\\ \\
    \noindent Compact objects are stars so dense that general relativity is required to describe their phenomenology. These are commonly known as neutron stars due to the fact that at very high densities matter is largely composed of neutrons. The matter inside these stars cannot be recreated in terrestrial laboratories and this makes them an ideal astrophysical laboratory to test theories of dense matter.
    
    Modelling the interior of neutron stars has required to overcome several theoretical challenges in nuclear physics. So much so that finding the equations of state that describes matter in such conditions is still one of the open problems in physics. Observations of neutron stars can constraint the theoretical models used to obtain the equation of state. But the lack of precision hasn't allowed to identify the correct theory, in such a way that fundamental characteristics as their composition are still up to debate.
    
    Here we propose using \quotes{physical acceptability} criteria (considerations of physical consistency and stability) to constraint existing ultradense matter models. In particular, the rigurous use of known stability criteria was emphasized. As a result, we found that all stelar models constructed with the equations of state considered do not satisfy a simple stability criterion against convective motion proposed recently by Hernández et al. \cite{Hernandez2018}.
    \end{spacing}
\end{abstract1}